\documentclass[12pt]{article}
\usepackage{amsmath}
\usepackage{graphicx}
\usepackage[usenames,dvipsnames]{xcolor}
\usepackage{tikz}
\begin{document}


\newcommand{\vct}[1]{\mathbf{#1}}
\newcommand{\vx}{\vct{x}}
\newcommand{\vxref}{\vct{x}^\mathrm{ref}}
\newcommand{\vy}{\vct{y}}
\newcommand{\vu}{\vct{u}}



\section{Problem}



Our problem is compute the Jacobian from a minimal deviation fitting.
%
Consider a reference structure $\vxref$ and a test structure $\vx_0$.
%
We can fit $\vxref$ to $\vx_0$ by rotation and translation.
%
The best fit structure is denoted as $\vy_0$.
%
Thus,
\[
  \vy_0 = \vct R \, \vxref + \vct T,
\]
where $\vct R$ and $\vct T$
are the rotation matrix and translation vector
that minimize $\| \vy_0 - \vx_0 \|$.


Since $\vct R$ and $\vct T$, and hence $\vy_0$,
depend on $\vx_0$,
we wish to compute the Jacobian
\[
J =
\left| \frac{\partial \vy_0 } { \partial \vx_0 } \right|,
\]
from the above best fitting procedure.
%
We wish to show that $J$ depends non-trivially on $\vx_0$.




\section{Numerical calculation}



The Jacobian can be numerically computed as follows.
%
First, we perturb the test structure a bit
as $\vx = \vx_0 + \delta \vx$,
%
We then rotate and translate $\vxref$ to get $\vy$
that best fit $\vx$.
%
The deviation of the fit structure is $\delta \vy = \vy - \vy_0$.
%
The idea is to compute the Jacobian $|\partial \vy/\partial \vx|$
from $\| \delta \vy \|$ and $\| \delta \vx \|$
after some averaging.



To implement this idea,
we use principal component analysis (PCA).
%
First,
we try different $\vx_i = \vx_0 + \delta \vx_i$, for $i = 1, \dots, M$,
where $\delta \vx_i$ is a zero-mean normal random vector
with the standard deviation of each component being $\sigma$.
%
For each $\vx_i$,
we can compute a $\vy_i$, hence $\delta \vy_i$ from fitting.
%
This allows us to compute the averaged covariance matrix, $\vct A$,
\begin{equation}
  \vct A = \frac{1}{M} \sum_{i = 1}^M \delta \vy_i \cdot \delta \vy_i^T.
\end{equation}
%
The dimension of the matrix is $3N \times 3N$
for a system of $N$ atoms.
%
The matrix $\vct A$ has only six nonzero eigenvalues,
$\lambda_1, \dots, \lambda_6$,
corresponding to the three translation and three rotation degrees of freedom.
%
The Jacobian is
\begin{equation}
J \equiv
\frac { 1 } { \sigma^{3N} }
\sqrt{ \prod_{i = 1}^6 \lambda_i }.
\end{equation}



\section{Results}



The above algorithm is implemented and for a dimer system,
with two equally-weighted atoms.
%
In the reference structure, $\vxref$,
the two atoms are separated by $2 \, l = 2$.
%
Our first test structure, $\vx_0^A$,
is identical to the reference structure
(but the perturbed structure $\vx^A$
won't be identical to $\vxref$ in general).
%
In the second test structure, $\vx_0^B$,
the two atoms are separated by $2 \, l = 4$.


From our calculations with $\sigma = 0.1$,
we get $\ln J^A = -2.304$, and $\ln J^B = -5.075$.
%
This clearly shows that the Jacobian depends on $\vx_0$.
But more important, we have
$J^B/J^A = 1/2^4$.
%
Can we explain this?




\section{Explanation}



We wish to show that $J$ depends non-trivially on $\vx_0$.
%
It is really shown that after a best fitting,
the translation vector $\vct T$
is precise the displacement between
the centers of mass of $\vx_0$ and $\vxref$.
%
Thus, we shall assume that the center of mass motion
is removed and consider only the rotation.
%
We wish to show $\vct R$ depends nontrivial on $\vct x_0$.



We shall place the two atoms on the $x$-axis,
with the center of mass fixed at the origin.
%
We observe that no matter how $\vx$ changes,
$\vy$ will be aligned along the axis of $\vx$
after the best fitting.
%
Thus, the ratio of $\delta \vy$ to $\delta \vx$
should be equal to the ratio of
the corresponding relative scales
of $\vy_0$ and $\vx_0$.
%
In our case,
it means
%
\[
  \frac { \delta \vy } { \delta \vx }
  = \frac { 1 } { l }.
\]
%
Now since $\vx_0^B$ is twice as large as $\vx_0^A$.
If $\delta \vx^B$ is of the same magnitude of $\delta \vx^A$,
then $\delta \vy^B$ would be only half as large as $\delta \vy^A$.
%
This is why we get a factor of $1/2$.



But $J_2/ J_1 = 1/2^4$.
%
How about the rest factor, $1/2^3$?
%
Well, we have three other $1/2$ factors
due to the translational degrees of freedom,
which follow a similar scaling argument.



\begin{figure}[h]
\begin{center}
  \newcommand{\sz}{1.5cm}
  \tikzstyle{ball} = [circle, minimum size=0.5*\sz]

  \begin{tikzpicture}
    \draw [black!30, thin] (-2.5*\sz, 0) -- (2.5*\sz, 0);
    \draw [black!50, thin, dashed] (-2.5*\sz, -0.75*\sz) -- (2.5*\sz, 0.75*\sz);

    \node[ball, fill=black!10!white] (xref1) at (-1.0*\sz, 0.0) [] {};
    \node[ball, fill=black!10!white] (xref2) at ( 1.0*\sz, 0.0) [] {};

    \node[ball, fill=black!20!white] (x01) at (-2.0*\sz, 0.0) [] {};
    \node[ball, fill=black!20!white] (x02) at ( 2.0*\sz, 0.0) [] {};

    \node[ball, draw=black!50, dashed] (y1) at (-1.0*\sz, -0.3*\sz) [] {};
    \node[ball, draw=black!50, dashed] (y2) at ( 1.0*\sz,  0.3*\sz) [] {};

    \node[ball, draw=black, dashed] (x1) at (-2.0*\sz, -0.6*\sz) [] {};
    \node[ball, draw=black, dashed] (x2) at ( 2.0*\sz,  0.6*\sz) [] {};

    %\path [draw] (x01) -- (x02);

    \path [draw] (xref1) -- (xref2);
    \draw [->, thick] (2.0*\sz, 0.0) -- (2.0*\sz, 0.6*\sz);
    \draw [->, thick] (1.0*\sz, 0.0) -- (1.0*\sz, 0.3*\sz);
    \draw [->, thick] (-2.0*\sz, 0.0) -- (-2.0*\sz, -0.6*\sz);
    \draw [->, thick] (-1.0*\sz, 0.0) -- (-1.0*\sz, -0.3*\sz);
  \end{tikzpicture}
\end{center}
\label{fig:dimer}
\end{figure}



\section{Discussion}


The above cases shows that the Jacobian is not a constant
because the difficult of adjusting the reference structure
in the minimal deviation fitting changes with configuration $\vx_0$.
%
If we go through the mathematics of computing $\vct R$,
this would be more easily understood.
%
We recall that in computing the best fitting,
each row $\vct R$ is a left eigenvector of the following $3\times 3$ matrix
%
\begin{equation}
\sum_{i = 1}^N \vx_{0i} \cdot (\vx_\mathrm{ref})^T + \vxref_i \cdot \vx_{0i}^T,
\end{equation}
where $i$ denotes the atom index.
%
Because the eigenvalue problem is quite complex and nonlinear,
it is unlikely that $\vct R$ depends linearly on $\vx_0$.


\end{document}

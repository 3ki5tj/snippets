\documentclass[12pt]{article}

\usepackage{amsmath}

\newcommand{\vct}[1]{\mathbf{#1}}
\providecommand{\vr}{} % clear \vr
\renewcommand{\vr}{\vct{r}}
\newcommand{\vk}{\vct{k}}
\newcommand{\vR}{\vct{R}}
\newcommand{\erf}{\mathrm{erf}}

\begin{document}

\section{Method}

\subsection{Integral equation}

The Ornstein-Zernike (OZ) relation is
\begin{equation}
  t(\vr) = \rho \, (c * h)(\vr)
         = \rho \, [c * (t + c)] (\vr),
  \label{eq:oz}
\end{equation}
%
where $(c * h)(\vr) = \int c(\vr') \, h(\vr - \vr') \, d\vr'$
is the convolution of $c(\vr)$ and $h(\vr)$.
%
In the Fourier space
\begin{equation}
  \tilde{t}(\vk) = \rho \, \tilde{c}(\vk) \, [\tilde{t}(\vk) + \tilde{c}(\vk)]
                 = \frac{ \rho \, \tilde{c}^2(\vk) }
                        { 1 - \rho \tilde{c}(\vk)    }.
  \label{eq:ozk}
\end{equation}

The closure is
\begin{equation}
  c(\vr) = \exp[-\beta \, \phi(\vr) + t(\vr) + B(\vr)] - t(\vr) - 1,
  \label{eq:closure}
\end{equation}
where $B(\vr)$ is the bridge function,
which vanishes in the hypernetted-chain (HNC) case.


\subsection{Long-range interaction}

For the electrostatic interaction,
$c(\vr)$ and $t(\vr)$ are long-ranged,
but their sum $h(\vr) = c(\vr) + t(\vr)$ is short-ranged.
%
Suppose we can identify the long-ranged component $\psi(\vr)$ in $t(\vr)$,
then the long-ranged component in $c(\vr)$ must be $-\psi(\vr)$.
%
Then we can define a new set of short-ranged functions
%
\begin{align}
  c_h(\vr)     &\equiv  c(\vr) + \psi(\vr), \label{eq:ch} \\
  t_h(\vr)     &\equiv  t(\vr) - \psi(\vr), \label{eq:th} \\
  \phi_h(\vr)  &\equiv  \phi(\vr) - \beta^{-1} \, \psi(\vr),
\end{align}
%
where $\psi(\vr)$ is the long-ranged component.


In terms of the new functions, we have
\begin{align}
  \tilde{t}_h(\vk) &= \frac{ \rho \, [ \tilde{c}_h(\vk) - \tilde{\psi}(\vk) ]^2 }
                           { 1 - \rho \, [\tilde{c}_h(\vk) - \tilde{\psi}(\vk) ] }
                      -\tilde{\psi}(\vk),
  \label{eq:ozk_short}
 \\
  c_h(\vr) &= \exp[-\beta \, \phi_h(\vr) + t_h(\vr) + B(\vr)] - t_h(\vr) - 1,
  \label{eq:closure_short}
\end{align}
%
Since now $c_h(\vr)$ and $t_h(\vr)$,
\eqref{eq:ozk_short} and \eqref{eq:closure_short}
can be solved more efficiently than
\eqref{eq:ozk} and \eqref{eq:closure}.



\subsection{Long-range component}

We should choose $\psi(\vr)$ such that $c_h(\vr)$ and $t_h(\vr)$ are short-ranged.
%
At a long distance, we have
\[
  c(\vr) \approx f(\vr)
         = e^{-\beta \phi(\vr)} - 1
         \approx -\beta \phi(\vr).
\]
Thus, in order for $c_h(\vr)$, defined in \eqref{eq:ch}, to vanish at a long distance,
we must have
\begin{equation}
  \psi(\vr) \approx \beta \phi(\vr)
            = \beta/r,
  \qquad \mbox{for $r \rightarrow \infty$}.
\label{eq:longrangelimit}
\end{equation}

In practice, we borrow the solution of Ewald sum, and set
\[
  \psi(\vr) = \beta \frac{ \erf\left[ r/(\sqrt{2} a) \right] }{ r }.
\]
The Fourier transform is known:
\[
  \tilde{\psi}(\vk) = \beta \frac{ 4 \pi \exp(-k^2 a^2/2) } { k^2 }.
\]
Since $\erf(x) \rightarrow 1$ for a large $x$,
\eqref{eq:longrangelimit} is satisfied.


\end{document}
